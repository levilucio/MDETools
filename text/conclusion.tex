\section{Conclusion}
\label{sec:conclusion}

We have presented in this article the lifecycle of the development of
a controller for a virtual rover, based on a controller for a real vehicle
developed at lab courses given by us. Our experience points to the fact that \af is
a sufficiently mature environment for developing embedded systems, in
particular controllers. The facilities for generating code for a specific platform (in our case a generic one) make life
for the developer of embedded code simple, as the communication infrastructure
with the underlying hardware can be fully automated. We have observed this
advantage when we, in the course of the lab courses, deployed the code
generated from models directly to Raspberry Pis, without any need for further customization. Additionally, the modularity enforced by \af
makes it easy to reuse parts of projects. We found that the copy/paste
facilities of \af are very helpful in that respect.

We have certainly encountered editing issues with \af's editor while building
the model for the challenge, but they were minor and the modelling experience
was very slighted affected by them. The calibration of controllers such as the
one we present in this paper also poses a problem, as it is mostly only
possible once the hardware is in the loop with the generated code. \af does not
provide a basic infrastructure for calibration (although a prototypical version
of such an infrastructure does exist). In practice, we have observed that a
significant amount of time still needs to be devoted to making sure the
parameters of the controller are well configured.
