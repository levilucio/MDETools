\section{Introduction}

\autofocus is a model-based development (MBD) environment for embedded systems,
based on the \textsf{Focus} theory~\cite{Broy:2001:SDI:374869}. \textsf{Focus} is a framework
encompassing computations supported by the notion of streams (“in particular
untimed, timed and time-synchronous streams”
\cite{Holzl:2007:AST:1927558.1927576}). The current version of \af follows a
string of earlier
prototypes~\cite{Holzl:2007:AST:1927558.1927576,DBLP:conf/models/AravantinosVTHS15}
started in 1996~\cite{Huber96autofocus--}. Existing literature on \af (some of
it published at MODELS) reports on particular aspects of the
tool~\cite{TMR2013,TMR2011,Lucio:17,DBLP:conf/se/VossEH14,Barner2016,Diewald2016,Carlan2017},
or on its application in the context of industrial case
studies~\cite{2009-a-top-down-methodology-for-the-development-of-automotive-software,2011KeylessEntry,Bohm:2014:FSE:2593850.2593856,DBLP:conf/models/AravantinosVTHS15,Barner2017,Eder2017}.
More information about current state of the \af-related research can be found in
the official site of the tool\footnote{ https://af3.fortiss.org/}. \af can be
freely downloaded and is open-source.

\af's goal is to demonstrate the feasibility and applicability of
MBD tooling approaches. The idea behind \af embraces seamless integration of all models throughout the
development process, encompassing requirements engineering on initial stage, system
modeling at a high level of abstraction, deployment and model simulation. \af
also comprises formal verification and testing. Being an open source tool with a 6 months
release period, \af embodies a study tool for proving scientific concepts and
methods which have been tested via industrial case studies.

In the context of domain-specific development, there are several approaches to
system modeling. The first of those is characterized by starting from a general
(non platform-specific) model and proceed by transforming this model into
specialized one. In this approach, domain specific languages are built by
restricting a universal language (such as the UML) and incarnated as development
tools. This is the “Model-Driven Architecture” concept and is represented by
such tools as Enterprise Architect~\cite{SparxSystems} or
Papyrus~\cite{Papyrus}. \af embodies a “bottom-up” approach, which aims at
guiding the modeler until full creation of domain-specific model.
In comparison with the former approach, \af follows the domain-specific modeling
philosophy, where only the strictly required concepts are developed into tools while starting
from a blank slate. The goal is to minimize the possibility of error by
enforcing, as much as possible, correctness-by-construction. Additionally, \af
is built on top of an extensible kernel which constitutes a base for further
development. Examples of other tools that follow \af's ``bottom-up'' approach are
Sirius\cite{Sirius} or JetBrains' MPS\cite{MPS}.

Tools that resemble \af in some way are Enterprise
Architect~\cite{SparxSystems}, Papyrus~\cite{Papyrus}, UML Designer~\cite{UMLDesigner}, Sirius\cite{Sirius}, JetBrains' MPS~\cite{MPS},
mbeddr~\cite{mbeddr} or Simulink~\cite{simulink}. Although space does not allow elaborating on the
differences between these tools and \af, our tool is to the best of our knowledge the
only open-source model-driven tool that supports the whole embedded-software development
cycle in an out-of-the-box, easily installable package. \af includes support
for requirements engineering, formal verification, deployment or domain space
exploration cases natively (among other features), which we have not found in
combined in one unique package in other tools.\\

In this paper we explain how we have developed an \af model to operate a rover
in a virtual environment in a way that it follows another leader rover that
advances freely, while always keeping a safe distance. ``Sensor" data regarding
the position of the rover in front and the distance between the two rovers is
provided by the environment. The rover has access to its own position and angle
of movement and should constantly adjust its speed and angle by directing power
to the wheels (turning implies asymetrically providing power to the left and
right wheels). The virtual environment the rover functions on has been proposed
is part of the MDE tool challenge proposed by the MDE Tools workshop at the
MODELS 2018 conference.

The paper is organized as follows. In \sect\ref{sec:toy_rover_controller} we
provide a high level description of a controller built in \af for a physical car in the
context of a lab course offered at the Technical University of Munich.
Section~\ref {sec:control_vr_model} then describes how we have adapted the
Adaptive Cruise Control component built during the lab courses to serve the
purpose of following the leader. In \sect\ref{sec:deploy_generate} we explain
how we have deployed the model and generated \clang code for the controller. Section
\sect\ref{sec:conclusion} concludes.
