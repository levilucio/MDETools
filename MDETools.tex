% This is samplepaper.tex, a sample chapter demonstrating the
% LLNCS macro package for Springer Computer Science proceedings;
% Version 2.20 of 2017/10/04
%
\documentclass[runningheads]{llncs}
%
\newcommand{\todo}[1]{\textbf{\textcolor{red}{TODO: #1}}} 

\usepackage{graphicx}

\usepackage{cite}
\usepackage{amsmath,amssymb,amsfonts}
\usepackage{algorithmic}
\usepackage{graphicx}
\usepackage{textcomp}
\usepackage{xspace}
\usepackage{url}
\def\BibTeX{{\rm B\kern-.05em{\sc i\kern-.025em b}\kern-.08em 
    T\kern-.1667em\lower.7ex\hbox{E}\kern-.125emX}}
    
% Put edit comments in a really ugly standout display
\usepackage{times}
 
% Macros for proof-reading & corrections
\usepackage[normalem]{ulem} % for \sout  
\usepackage{xcolor} 

% Macros for proof-reading & corrections
\usepackage[normalem]{ulem} % for \sout  
\usepackage{xcolor} 

\usepackage{ifthen}
\usepackage{amssymb} 
\newboolean{showcomments}  
\setboolean{showcomments}{true} % toggle to show or hide comments
\ifthenelse{\boolean{showcomments}}     
  {\newcommand{\nb}[2]{
    \fcolorbox{gray}{yellow}{\bfseries\sffamily\scriptsize#1}
    {$\blacktriangleright$#2$\blacktriangleleft$}
   }
   \newcommand{\version}{\emph{\scriptsize$-$working$-$}} 
  }
  {\newcommand{\nb}[2]{}
   \newcommand{\version}{} 
  } 

\newcommand\levi[1]{\nb{Levi}{\textcolor{teal}{#1}}}
\newcommand\andreas[1]{\nb{Andreas}{\textcolor{teal}{#1}}}
\newcommand\sk[1]{\nb{sk}{\textcolor{teal}{#1}}}

\newcommand{\af}{\textsc{AF3}\xspace}
\newcommand{\autofocus}{\textsc{AutoFOCUS3}\xspace}

\newcommand{\ears}{\textsc{Ears}\xspace} 
\newcommand{\etal}{\emph{et al.}\xspace}
\newcommand{\earsctrl}{\textsc{Ears-Ctrl}\xspace}
\newcommand{\ltl}{\textsc{Ltl}\xspace}
\newcommand{\iets}{\textsc{Iets3}\xspace}
\newcommand{\fig}{\text{Figure}\xspace}
\newcommand{\sect}{\text{section}\xspace} 
\newcommand{\tab}{\text{table}\xspace}
\newcommand{\eg}{\text{e.g.}\xspace}
\newcommand{\pid}{\textsc{Pid}\xspace}
\newcommand{\clang}{\textsc{C}\xspace}
\newcommand{\java}{\textsc{Java}\xspace}
\newcommand{\ecu}{\textsc{Ecu}\xspace}
\newcommand{\acc}{\textsc{Acc}\xspace} 

% Used for displaying a sample figure. If possible, figure files should
% be included in EPS format.
%
% If you use the hyperref package, please uncomment the following line
% to display URLs in blue roman font according to Springer's eBook style:
% \renewcommand\UrlFont{\color{blue}\rmfamily}

\begin{document} 
% 
\title{Controlling a Virtual Rover using AutoFOCUS3}
%
%\titlerunning{Abbreviated paper title}
% If the paper title is too long for the running head, you can set
% an abbreviated paper title here
%
\author{Levi L\'ucio \and
Sudeep Kanav \and Andreas Bayha \and Johannes Eder} 
%
%\authorrunning{Levi L\'ucioSudeep Kanaav}
% First names are abbreviated in the running head.
% If there are more than two authors, 'et al.' is used.
%
\institute{fortiss GmbH\\ Guerickestra\ss e 25\\80805 M\"unchen\\
\email{\{lucio,kanav,bayha,eder\}@fortiss.org}}
%
\maketitle              % typeset the header of the contribution
%
\begin{abstract}
\autofocus (\af) is a mature model-driven engineering environment for developing
software for embedded systems. For the past 20 years, several versions of \af
have served as a platform for experimenting with cutting edge research ideas in
Model-Driven Development.
\af is a tool that fully encompasses the software lifecycle, from requirements,
to architecture, simulation, deployment, code generation and verification. In
this article we describe how we used an existing model of a complex controller
for a real-life miniature vehicle and have downsized and adapted it to control a rover in a
virtual environment. The model we present here automates the maneuvering
of a rover to follow another leader rover in a virtual environment, while
keeping a safe distance to it. The controller operates by adapting the rover's
speed and steering according to the position and movements of the leader.
The results we present in this article illustrate the whole development cycle of
an embedded system using \af, from the development of the model down to
deployment to a specific platform as well as code generation and connecting to
the hardware. 
 
\keywords{Modelling Environment \and Embedded Systems \and Deployment \and Code
Generation \and Controller.}
\end{abstract}
  
\section{introduction}

Based on the FOCUS theory~\cite{Broy:2001:SDI:374869}, a framework encompassing
computations supported by the notion of streams (“in particular untimed, timed
and time-synchronous streams” \cite{Holzl:2007:AST:1927558.1927576}),
the current version of \af follows a string of earlier
prototypes~\cite{Holzl:2007:AST:1927558.1927576,DBLP:conf/models/AravantinosVTHS15}, started in 1996~\cite{Huber96autofocus--}. Existing literature on \af
(some of it published at MODELS) reports on particular aspects of the
tool~\cite{TMR2013,TMR2011,Lucio:17,DBLP:conf/se/VossEH14,Barner2016,Diewald2016,Carlan2017}, or on its
application in the context of industrial case
studies~\cite{2009-a-top-down-methodology-for-the-development-of-automotive-software,2011KeylessEntry,Bohm:2014:FSE:2593850.2593856,DBLP:conf/models/AravantinosVTHS15,Barner2017,Eder2017}.
More information about current state of the \af-related research can be found in
the official site of the tool\footnote{ https://af3.fortiss.org/}. \af can be
freely downloaded and is open-source.

\af is an academic implementation of a model-based development (MBD)
environment for embedded systems. Its goal is to demonstrate the feasibility and applicability of
MBD tooling approaches. The idea behind \af embraces seamless integration of all models throughout the
development process, encompassing requirements engineering on initial stage, system
modeling at a high level of abstraction, deployment and model simulation. \af
also comprises formal verification and testing. Being an open source tool with a 6 months
release period, \af embodies a study tool for proving scientific concepts and
methods which have been tested via industrial case studies.

In the context of domain-specific development, there are several approaches to
system modeling. The first of those is characterized by starting from a general
(non platform-specific) model and proceed by transforming this model into
specialized one. In this approach, domain specific languages are built by
restricting a universal language (such as the UML) and incarnated as development
tools. This is the “Model-Driven Architecture” concept and is represented by
such tools as Enterprise Architect~\cite{SparxSystems} or
Papyrus~\cite{Papyrus}. \af embodies a “bottom-up” approach, which aims at
guiding the modeler until full creation of domain-specific model.
In comparison with the former approach, \af follows the domain-specific modeling
philosophy, where only the strictly required concepts are developed into tools while starting
from a blank slate. The goal is to minimize the possibility of error by
enforcing, as much as possible, correctness-by-construction. Additionally, \af
is built on top of an extensible kernel which constitutes a base for further
development. Examples of other tools that follow \af's ``bottom-up'' approach are
Sirius\cite{Sirius} or JetBrains' MPS\cite{MPS}.

Tools that resemble \af in some way are Enterprise
Architect~\cite{SparxSystems}, Papyrus~\cite{Papyrus}, UML Designer~\cite{UMLDesigner}, Sirius\cite{Sirius}, JetBrains' MPS~\cite{MPS},
mbeddr~\cite{mbeddr} or Simulink~\cite{simulink}. Although space does not allow elaborating on the
differences between these tools and \af, our tool is to the best of our knowledge the
only open-source model-driven tool that supports the whole embedded-software development
cycle in an out-of-the-box, easily installable package. \af includes support
for requirements engineering, formal verification, deployment or domain space
exploration cases natively (among other features), which we have not found in
combined in one unique package in other tools. 

\section{Controlling a Physical Vehicle with \autofocus}
\label{sec:toy_rover_controller}

The model that we developed for the MDE tool challenge is based on a larger
model, that was originally created in the context of a lab course at Technical
University Munich. Two subsequent courses involving 10 students built not only 
the logical model for the vehicle, but also the hardware platform of a car in
a scale of 1:10.
An important requirement for this vehicle was a high level of realism. Accordingly,
a professional platform with a realistic Ackermann steering and electrical all
wheel drive was chosen and configured in such a way that the driving
dynamics correspond to a realistic car. A picture of the vehicle is shown in
\fig\ref{fig:vehicle}.

\begin{figure}[!h]
\centering
\includegraphics[width=1\textwidth]{images/FullSizeRender.jpg}
\caption{A Physical Rover used for a lab course at the Technical University
of Munich}
\label{fig:vehicle}
\end{figure}

The model that the students developed in the first lab course implemented basic
driving functionalities such as steering, braking, accelerating, gear shifting
and different drive modes, as well as two driver assistance functions for
emergency braking and adaptive cruise control.
The second lab course extended this outcome with lane keeping as well as
vehicle2vehicle communication and platooning.
Our main interest in the development of such a vehicle was to show the
applicability of model-based development by using our tool \autofocus on the one
hand, and on the other hand to come up with a software architecture for future
(semi-) autonomous cars. The general question we addressed was
how (semi-) autonomous functionality can be integrated into a software
architecture of a car, using a model-based approach. The current state of the
architecture is shown in \fig\ref{fig:vehicle_architecture} and is given as a
means to illustrate the complexity of the model of the controller developed by
the students.

\begin{figure}[!h]
\centering
\includegraphics[width=1.4\textwidth, angle=90]{images/ACC_architecture}
\caption{The \af Model developed by the Students to Control a Physical Vehicle}
\label{fig:vehicle_architecture}
\end{figure}

For the MDE tool challenge we used the Adaptive Cruise Control (\acc) part of
the model, highlighted in  \fig\ref{fig:vehicle_architecture}. Because of the
component based approach of \af, we were able to reuse the component which
realized this function and adapt it to the challenge by developing against the
component's interface. Although we kept that part of the functionality that
adapts the distance to the leader rover, we had to incorporate in the model new
capabilities to allow automatic steering in order to implement the ``follow the
leader'' requirement. Naturally, we also had to adapt the inputs and the outputs of the
component to the data provided and expected by the virtual environment.


%\section{Reducing the Model to Control the Virtual Rover}  
 
\section{Controlling the Virtual Rover NEW}
\label{sec:control_vr_model}

\begin{figure}[!h]
	\centering
	\includegraphics[width=1\textwidth]{images/acc.png}
	\caption{The controller for the Virtual Rover}
	\label{fig:acc}
\end{figure}

In \fig\ref{fig:acc} we depict the top-level model of the controller for the
follower vehicle. The controller is meant to operate in a loop by reading the
 {distance} to
the leader rover, the GPS coordinates of the leader (\textit{LeaderPosition})
and the rover's own (\textit{RoverPosition}) GPS coordinates as well as its own
orientation with respect to the north (\textit{RoverAngle}).
Note that the inputs to the model appear in \fig\ref{fig:acc}  as small black
circles, while the outputs have the same shape but are white. The power provided
to the wheels is constantly updated to reflect the changes in the input values
to the controller.

The controller for the virtual rover is composed by three \af components,
as explained in the next sections.

\subsection{Component StraightPower}
The \textsf{StraightPower} component is responsible for calculating the required
 forward power based on the distance to the leader.

This component is composed of two components as shown in \fig\ref{fig:straight}.
The component \textit{CalculateDistanceError} calculates the \textit{error} with
respect to the ideal distance with the leader. For the proposed challenge, the
follower was required to remain in the distance range between 12 and 15 from the
leader. We have thus taken the ideal distance as the average of these two
values, i.e. 13.5. This is a constant and can be easily changed to allow for different
ranges.

\begin{figure}[!h]
	\centering
	\includegraphics[width=1\textwidth]{images/straight.png}
	\caption{Subcomponents implementing the StraightAcceleration component.}
	\label{fig:straight}
\end{figure}


We then use this \textit{error} and feed it to a PID controller for calculating
the power to be directed forward. The general equation for a PID controller
is:
\begin{equation}
u = K_Pe + K_II + K_DD
\label{formula:pidController}
\end{equation}
, where $K_P, K_I$ and $K_D$ are the parameters of the controller, $e$ is the error with the desired value, $I$ is the integral - summation of the previous errors, and $D$ the differential - difference with the last error.

For calculating the forward power we have used the following constant values: $K_P = 5$, $K_I=1.5$, $K_D=30$.

\subsection{Component RotationalDifferential}

The rover turns when the left and right wheels rotate at different speeds. The
magnitude of the difference is proportional to the turning angle.

When the leader turns the follower also has to turn in order to follow the
leader. In order to achieve this the \textsf{RotationalDifferential} component
calculates the required difference between the power applied to the right and
the left wheel to turn the rover to provide the correct turning angle.

The component \textsf{bearingAngle} calculates the bearing of the leader with
respect to north when seen from the follower. This calculation uses the GPS
positions of the follower and the leader. We then calculate the
\textit{angleError} i.e., the difference between the orientation of the follower
(with respect to north) and the bearing angle. This
\textit{angleError} is then passed onto another PID controller in order to
calculate the required difference in power sent to the rover's right and left
wheels. The sign of this value decides the direction of the turning.

\begin{figure}[!h]
	\centering
	\includegraphics[width=1\textwidth]{images/rotation.png}
	\caption{Subcomponents implementing the RotationalDifferential component.}
	\label{fig:rotation}
\end{figure}

The constants used of the PID controller (equation
\ref{formula:pidController}) for calculating the rotational differential are:
$K_P = 2$, $K_I=0.75$, $K_D=10$.

\subsection{Component CalculateFinalPower}
The \textsf{CalculateFinalPower} component takes the forward power and rotational differential, and outputs the final power to apply to the
right and left wheels.
In addition to calculating the values for the right and left power, the
component also normalizes the amount of power provided in case the calculated
value exceeds the maximum.

The environment of the rover challenge proposed by the MDETools workshop
provides at the end of a run of the system, which lasts one minute, the percentage of time during which the rover was within the
expected distance limits. The system we developed consistently
stays within these limits over 70\% of the runs we have attempted. Although we
have not tuned the values of the PID controller further, we believe even better results could be achieved.
The \af models we have used for the challenge can be downloaded
at~\cite{af3_mdetools}. For readers interested in further experimentation,
instructions accompanying the model provide the steps on how to install and
deploy the software.

 
\section{Deployment and Code Generation}
\label{sec:deploy_generate}

After the model is built, it needs to be deployed on an architecture. For the
real rover mentioned in \sect~\ref{sec:toy_rover_controller} the architecture
is a Raspberry Pi that can connect to the sensors and actuators of the device.

The virtual rover simulation environment used in the context of this article
communicates using TCP ports. Additionally, the signals flowing from the virtual
environment and back are different from the ones for the real rover. For
instance, the real rover accepts \emph{target speed} as input and the hardware
of the rover itself controls engine power (using an embedded \pid controller) in
order to attain such a speed and maintain it. The virtual rover expects that
power to the wheels is provided as a means to attain a certain speed.

\af provides a generic, non-device specific architecture for deployment, as
shown in \fig~\ref{fig:deployment_general}.

\begin{figure}[!h]
\centering
\includegraphics[width=.8\textwidth]{images/deployment_general.png}
\caption{A Generic ECU for the Virtual Rover Controller}
\label{fig:deployment_general}
\end{figure}

Additionally, the ports of the of the \ecu need to be mapped to the logical
ports of the controller of the model we have defined in
\sect~\ref{sec:control_vr_model}, as depicted in
figure~\ref{ig:deployment_ports}.

\begin{figure}[!h]
\centering
\includegraphics[width=1\textwidth]{images/deployment_ports_mapping.png}
\caption{Deploying the Logical Ports onto the \ecu ports}
\label{fig:deployment_ports}
\end{figure}
 
Deploying to an architecture provides the skeleton of an interface that declares
the signatures of the methods that are used by the controller logic to
communicate with the device underneath. When the architecture is fully defined,
then the glue code with the device can also be automatically generated. For our
work we have deployed onto a generic architecture as a means to automatically
generate the structure of our controller's communication infrastructure as \clang\footnote{Besides \clang, \af also allows the generating \java code.} code. The logic corresponding to the model we have presented in
\sect~\ref{sec:toy_rover_controller} is also generated as \clang code and is
meant to run in a loop with the controlled device, in this case the virtual
rover.

The \clang code that is generated for the generic architecture only provides the
interface for the functions that read the sensors and send commands on the
actuators of the virtual environment. Because of that, a manual step of
coding such methods and connecting the controller with the virtual environment
via TCP was additionally necessary to connect the controller to the rover and to
finalize the deployment of the model onto the hardware.




\section{Conclusion}
\label{sec:conclusion}

We have presented in this article the lifecycle of the development of
a controller for a virtual rover, based on a controller for a real vehicle
developed at lab courses given by us. Our experience points to the fact that \af is
a sufficiently mature environment for developing embedded systems, in
particular controllers. The facilities for generating code for a specific platform (in our case a generic one) make life
for the developer of embedded code simple, as the communication infrastructure
with the underlying hardware can be fully automated. We have observed this
advantage when we, in the course of the lab courses, deployed the code
generated from models directly to Raspberry Pis, without any need for further customization. Additionally, the modularity enforced by \af
makes it easy to reuse parts of projects. We found that the copy/paste
facilities of \af are very helpful in that respect.

We have certainly encountered editing issues with \af's editor while building
the model for the challenge, but they were minor and the modelling experience
was very slighted affected by them. The calibration of controllers such as the
one we present in this paper also poses a problem, as it is mostly only
possible once the hardware is in the loop with the generated code. \af does not
provide a basic infrastructure for calibration (although a prototypical version
of such an infrastructure does exist). In practice, we have observed that a
significant amount of time still needs to be devoted to making sure the
parameters of the controller are well configured.
 

\bibliographystyle{abbrv} 
\bibliography{./references}

\end{document}
